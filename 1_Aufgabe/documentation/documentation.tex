\documentclass{article}

% Seitenformat festlegen
\usepackage[a4paper]{geometry}

% Bibliotheken für deutsche Sprache
\usepackage{ucs}
\usepackage[utf8x]{inputenc}
\usepackage[T1]{fontenc}
\usepackage[ngerman]{babel}

% Bibliotheken für mathematische Symbole
\usepackage{amsmath,amssymb}


\newcommand{\tabspace}{\phantom{} \hspace{25pt}}
\newcommand{\linespace}{\phantom{} \vspace{10pt}}
\newcommand{\linespacesmall}{\phantom{} \vspace{5pt}}
\newcommand{\eqspa}{\phantom{} \hspace{10pt}}



\begin{document}

\begin{flushleft}


\textbf{Darstellung der Personen} \newline
Wir verfolgen in unserem Ansatz eine Zuweisung von Personen auf die drei möglichen Positionen.
Damit sind die einzelnen Positionen Variablen dar, welche Werte aus der Menge der Personen enthalten.
Anders ausgedrückt ist die Menge der Personen $Personen$ die Domäne für genau diese Variablen: \newline
$
Personen = \{ albert, berta, chris, detlef, emma, felix \} \newline
$



\linespace
\textbf{Darstellung der Positionen} \newline
Zur Darstellung der Positionen definieren wir folgende Prädikate: \newline
$vorsitzender(X)$: X ist Vorsitzender \newline
$stellvertreter(X)$: X ist Stellvertreter \newline
$sekretaer(X)$: X ist Sekretär

\linespacesmall
Zur Darstellung der Zugehörigkeit einer Person $X$ zur Leitung definieren wir das Prädikat $leitung(X)$: \newline
$
\forall X \in Personen: 
     vorsitzender(X) \lor
     stellvertreter(X) \lor 
     sekretaer(X) 
     \to leitung(X)
$

\linespacesmall
Zur Darstellung dass eine Person $X$ über einer Person $Y$ steht definieren wird das Prädikat $stehtueber(X, Y)$: \newline
$
\forall X, Y \in Personen: 
     (leitung(X) \land \neg leitung(Y)) \lor \newline
     \tabspace (vorsitzender(X) \land stellvertreter(Y)) \lor \newline
     \tabspace (vorsitzender(X) \land sekretaer(Y)) \lor \newline
     \tabspace (stellvertreter(X) \land sekretaer(Y)) 
     \to stehtueber(X, Y) \newline
$



\linespace
\textbf{Darstellung der einschränkenden Regeln} \newline
Im Folgenden stellen wir die einschränkenden Regeln dar. \newline
Zur Umsetzung der Regeln in Prolog formen wir diese zuerst in die konjunktive Normalform (KNF) um. Dazu nutzen wir die Definition der Implikation und die Umformungsregeln für Doppelnegation, Distributivgesetz und DeMorgan. \newline
Zum späteren Testen der Regeln nutzen wir einerseits für positive Tests genau diese KNF. Anderseits nutzen wir eine Negation der Regel (welche wir ebenfalls in eine KNF umformen), um auch negative Tests durchzuführen. \newline



\pagebreak
\textbf{a) } Albert will nicht zur Leitung gehören, wenn Emma nicht Vorsitzende wird.
$
\newline
\eqspa \neg vorsitzender(emma) \to \neg leitung(albert) \newline
\equiv \neg \neg vorsitzender(emma) \lor \neg leitung(albert) \newline
\equiv vorsitzender(emma) \lor \neg leitung(albert) \textbf{ (KNF)} \newline
\newline
\textbf{Negation:} \newline
\eqspa \neg (vorsitzender(emma) \lor \neg leitung(albert)) \newline
\equiv \neg vorsitzender(emma) \land \neg \neg leitung(albert) \newline
\equiv \neg vorsitzender(emma) \land leitung(albert) \textbf{ (KNF)} \newline
$



\linespace
\textbf{b) } Berta will nicht zur Leitung gehören, wenn sie über Chris stehen soll.
$
\newline
\eqspa stehtueber(berta, chris) \to \neg leitung(berta) \newline
\equiv \neg stehtueber(berta, chris) \lor \neg leitung(berta) \textbf{ (KNF)} \newline
\newline
\textbf{Negation:} \newline
\eqspa \neg (\neg stehtueber(berta, chris) \lor \neg leitung(berta)) \newline
\equiv \neg \neg stehtueber(berta, chris) \land \neg \neg leitung(berta) \newline
\equiv stehtueber(berta, chris) \land leitung(berta) \textbf{ (KNF)} \newline
$



\linespace
\textbf{c) } Berta will unter gar keinen Umständen zusammen mit Felix arbeiten.
$
\newline
\eqspa \neg (leitung(berta) \land leitung(felix)) \newline
\equiv \neg leitung(berta) \lor \neg leitung(felix) \textbf{ (KNF)} \newline
\newline
\textbf{Negation:} \newline
\eqspa \neg (\neg leitung(berta) \lor \neg leitung(felix)) \newline
\equiv \neg \neg leitung(berta) \land \neg \neg leitung(felix) \newline
\equiv leitung(berta) \land leitung(felix) \textbf{ (KNF)} \newline
$



\linespace
\textbf{d) } Chris will nicht mitarbeiten, wenn der Leitung Emma und Felix zusammen angehören.
$
\newline
\eqspa leitung(emma) \land leitung(felix) \to \neg leitung(chris) \newline
\equiv \neg (leitung(emma) \land leitung(felix)) \lor \neg leitung(chris) \newline
\equiv \neg leitung(emma) \lor \neg leitung(felix) \lor \neg leitung(chris) \textbf{ (KNF)} \newline
\newline
\textbf{Negation:} \newline
\eqspa \neg (\neg leitung(emma) \lor \neg leitung(felix) \lor \neg leitung(chris)) \newline
\equiv \neg \neg leitung(emma) \land \neg \neg leitung(felix) \land \neg \neg leitung(chris) \newline
\equiv leitung(emma) \land leitung(felix) \land leitung(chris) \textbf{ (KNF)} \newline
$



\pagebreak
\textbf{e) } Chris wird nicht mitarbeiten, wenn Felix Vorsitzender oder Berta Sekretär ist.
$
\newline
\eqspa vorsitzender(felix) \land sekretaer(berta) \to \neg leitung(chris) \newline
\equiv \neg (vorsitzender(felix) \land sekretaer(berta)) \lor \neg leitung(chris) \newline
\equiv \neg vorsitzender(felix) \lor \neg sekretaer(berta) \lor \neg leitung(chris) \textbf{ (KNF)} \newline
\newline
\textbf{Negation:} \newline
\eqspa \neg (\neg vorsitzender(felix) \lor \neg sekretaer(berta) \lor \neg leitung(chris)) \newline
\equiv \neg \neg vorsitzender(felix) \land \neg \neg sekretaer(berta) \land \neg \neg leitung(chris) \newline
\equiv vorsitzender(felix) \land sekretaer(berta) \land leitung(chris) \textbf{ (KNF)} \newline
$



\linespace
\textbf{f) } Detlef wird nicht mit Chris oder Emma arbeiten, wenn er dem einen oder anderen unterstellt ist.
$
\newline
\eqspa stehtueber(chris, detlef) \lor stehtueber(emma, detlef) \to \neg leitung(detlef) \newline
\equiv \neg (stehtueber(chris, detlef) \lor stehtueber(emma, detlef)) \lor \neg leitung(detlef) \newline
\equiv (\neg stehtueber(chris, detlef) \land \neg stehtueber(emma, detlef)) \lor \neg leitung(detlef) \newline
\equiv (\neg stehtueber(chris, detlef) \lor \neg leitung(detlef)) \land (\neg stehtueber(emma, detlef) \lor \neg leitung(detlef)) \textbf{ (KNF)} \newline
\newline
\textbf{Negation:} \newline
\eqspa \neg ((\neg stehtueber(chris, detlef) \lor \neg leitung(detlef)) \land (\neg stehtueber(emma, detlef) \lor \neg leitung(detlef))) \newline
\equiv \neg (\neg stehtueber(chris, detlef) \lor \neg leitung(detlef)) \lor \neg (\neg stehtueber(emma, detlef) \lor \neg leitung(detlef)) \newline
\equiv (\neg \neg stehtueber(chris, detlef) \land \neg \neg leitung(detlef)) \lor (\neg \neg stehtueber(emma, detlef) \land \neg \neg leitung(detlef)) \newline
\equiv (stehtueber(chris, detlef) \land leitung(detlef)) \lor (stehtueber(emma, detlef) \land leitung(detlef)) \newline
\equiv (stehtueber(chris, detlef) \lor stehtueber(emma, detlef)) \land leitung(detlef) \textbf{ (KNF)} \newline
$



\linespace
\textbf{g) }  Emma will nicht Stellvertreter sein.
$
\newline
\eqspa \neg stellvertreter(emma) \textbf{ (KNF)} \newline
\newline
\textbf{Negation:} \newline
\eqspa \neg \neg stellvertreter(emma) \newline
\equiv stellvertreter(emma) \textbf{ (KNF)} \newline
$



\linespace
\textbf{h) } Emma will nicht Sekretär sein, wenn Detlef Mitglied der Leitung ist.
$
\newline
\eqspa leitung(detlef) \to \neg sekretaer(emma) \newline
\equiv \neg leitung(detlef) \lor \neg sekretaer(emma) \textbf{ (KNF)} \newline
\newline
\textbf{Negation:} \newline
\eqspa \neg (\neg leitung(detlef) \lor \neg sekretaer(emma)) \newline
\equiv \neg \neg leitung(detlef) \land \neg \neg sekretaer(emma) \newline
\equiv leitung(detlef) \land sekretaer(emma) \textbf{ (KNF)} \newline
$



\linespace
\textbf{i) } Emma will nicht zusammen mit Albert arbeiten, wenn Felix nicht der Leitung angehört.
$
\newline
\eqspa \neg leitung(felix) \to \neg (leitung(emma) \land leitung(albert)) \newline
\equiv \neg \neg leitung(felix) \lor \neg (leitung(emma) \land leitung(albert)) \newline
\equiv leitung(felix) \lor \neg leitung(emma) \lor \neg leitung(albert) \textbf{ (KNF)} \newline
\newline
\textbf{Negation:} \newline
\eqspa \neg (leitung(felix) \lor \neg leitung(emma) \lor \neg leitung(albert)) \newline
\equiv \neg leitung(felix) \land \neg \neg leitung(emma) \land \neg \neg leitung(albert) \newline
\equiv \neg leitung(felix) \land leitung(emma) \land leitung(albert) \textbf{ (KNF)} \newline
$



\linespace
\textbf{j) } Felix will nur mitarbeiten, wenn er oder Chris Vorsitzender wird.
$
\newline
\eqspa leitung(felix) \to vorsitzender(felix) \lor vorsitzender(chris) \newline
\equiv \neg leitung(felix) \lor vorsitzender(felix) \lor vorsitzender(chris) \textbf{ (KNF)} \newline
\newline
\textbf{Negation:} \newline
\eqspa \neg (\neg leitung(felix) \lor vorsitzender(felix) \lor vorsitzender(chris)) \newline
\equiv \neg \neg leitung(felix) \land \neg vorsitzender(felix) \land \neg vorsitzender(chris) \newline
\equiv leitung(felix) \land \neg vorsitzender(felix) \land \neg vorsitzender(chris) \textbf{ (KNF)} \newline
$



\end{flushleft}

\end{document}
