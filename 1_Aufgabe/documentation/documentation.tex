\documentclass{article}

% Seitenformat festlegen
\usepackage[a4paper]{geometry}

% Bibliotheken für deutsche Sprache
\usepackage{ucs}
\usepackage[utf8x]{inputenc}
\usepackage[T1]{fontenc}
\usepackage[ngerman]{babel}

% Bibliotheken für mathematische Symbole
\usepackage{amsmath,amssymb}


\newcommand{\tabspace}{\phantom{} \hspace{25pt}}
\newcommand{\linespace}{\phantom{} \vspace{10pt}}
\newcommand{\eqspa}{\phantom{} \hspace{10pt}}



\begin{document}

\begin{flushleft}

$
\newline
Personen = \{ albert, berta, chris, detlef, emma, felix \}
$

$
\newline
\forall X \in Personen: 
     vorsitzender(X) \lor
     stellvertreter(X) \lor 
    sekretaer(X) 
     \to leitung(X) \newline
\forall X, Y \in Personen: 
     (leitung(X) \land \neg leitung(Y)) \lor 
     (vorsitzender(X) \land stellvertreter(Y)) \lor \newline
     \tabspace (stellvertreter(X) \land sekretaer(Y)) 
     \to stehtueber(X, Y) \newline
$



\linespace
\textbf{a) } Albert will nicht zur Leitung gehören, wenn Emma nicht Vorsitzende wird.
$
\newline
\eqspa \neg vorsitzender(emma) \to \neg leitung(albert) \newline
\equiv \neg \neg vorsitzender(emma) \lor \neg leitung(albert) \newline
\equiv vorsitzender(emma) \lor \neg leitung(albert) \textbf{ (KNF)} \newline
\newline
\textbf{Negation:} \newline
\eqspa \neg(vorsitzender(emma) \lor \neg leitung(albert)) \newline
\equiv \neg vorsitzender(emma) \land \neg \neg leitung(albert) \newline
\equiv \neg vorsitzender(emma) \land leitung(albert) \textbf{ (KNF)} \newline
$



\linespace
\textbf{b) } Berta will nicht zur Leitung gehören, wenn sie über Chris stehen soll.
$
\newline
\eqspa stehtueber(berta, chris) \to \neg leitung(berta) \newline
\equiv \neg stehtueber(berta, chris) \lor \neg leitung(berta) \textbf{ (KNF)} \newline
\newline
\textbf{Negation:} \newline
\eqspa \neg(\neg stehtueber(berta, chris) \lor \neg leitung(berta)) \newline
\equiv \neg \neg stehtueber(berta, chris) \land \neg \neg leitung(berta) \newline
\equiv stehtueber(berta, chris) \land leitung(berta) \textbf{ (KNF)} \newline
$



\linespace
\textbf{c) } Berta will unter gar keinen Umständen zusammen mit Felix arbeiten.
$
\newline
\eqspa \neg(leitung(berta) \land leitung(felix)) \newline
$



\linespace
\textbf{d) } Chris will nicht mitarbeiten, wenn der Leitung Emma und Felix zusammen angehören.
$
\newline
\eqspa leitung(emma) \land leitung(felix) \to \neg(leitung(chris)) \newline
$



\linespace
\textbf{e) } Chris wird nicht mitarbeiten, wenn Felix Vorsitzender oder Berta Sekretär ist.
$
\newline
\eqspa vorsitzender(felix) \land sekretaer(berta) \to \neg(leitung(chris)) \newline
$



\linespace
\textbf{f) } Detlef wird nicht mit Chris oder Emma arbeiten, wenn er dem einen oder anderen unterstellt ist.
$
\newline
\eqspa steht\_ueber(chris, detlef) \lor steht\_ueber(emma, detlef) \to \neg(leitung(detlef)) \newline
$



\linespace
\textbf{g) }  Emma will nicht Stellvertreter sein.
$
\newline
\eqspa \neg(stellvertreter(emma)) \newline
$



\linespace
\textbf{h) } Emma will nicht Sekretär sein, wenn Detlef Mitglied der Leitung ist.
$
\newline
\eqspa leitung(detlef) \to \neg(sekretaer(emma)) \newline
$



\linespace
\textbf{i) } Emma will nicht zusammen mit Albert arbeiten, wenn Felix nicht der Leitung angehört.
$
\newline
\eqspa \neg(leitung(felix)) \to \neg(zusammenarbeiten(emma, albert)) \newline
$



\linespace
\textbf{j) } Felix will nur mitarbeiten, wenn er oder Chris Vorsitzender wird.
$
\newline
\eqspa leitung(felix) \leftrightarrow vorsitzender(felix) \lor vorsitzender(chris)
$

\end{flushleft}

\end{document}
